\newif \ifshare
% \sharetrue % Comment this if you want animation
\ifshare % "Share mode" without animation.
\documentclass[table, trans, aspectratio = 169]{beamer}
\else % "Presentation mode" with animation.
\documentclass[table, aspectratio = 169]{beamer}
\fi
\usepackage[T1]{fontenc}
\usepackage{color}
\usepackage{graphicx}
\usepackage{subfig}
\usepackage{diagbox}
\usepackage{amssymb}% http://ctan.org/pkg/amssymb
\usepackage{pifont}% http://ctan.org/pkg/pifont
\newcommand{\cmark}{\ding{51}}%
\newcommand{\xmark}{\ding{55}}%

% \pdfsuppresswarningpagegroup=1

\let\pgfmathMod=\pgfmathmod\relax

\graphicspath{{../figs/PDF/}}

\usetheme{Boadilla}

\definecolor{struct}{HTML}{f72e5c}
\definecolor{alert}{HTML}{f44336}
\definecolor{example}{HTML}{aeea00}
\definecolor{good}{HTML}{8bc34a}
\definecolor{notgoodnotbad}{HTML}{ff9800}

\setbeamercolor{structure}{fg = struct}
\setbeamercolor{normal text}{fg = white, bg = black}
\setbeamercolor{example text}{fg = example}
\setbeamercolor{alerted text}{fg = alert}
\setbeamercolor{footline}{fg = white}

\title[CiliumCon NA 2023]{Adding \texttt{\_\_kconfig} Support to Cilium/eBPF}
\subtitle{CiliumCon NA 2023}
\author[Francis Laniel (\texttt{flaniel@linux.microsoft.com})]{Francis Laniel\\\texttt{flaniel@linux.microsoft.com}}
\date{6th November 2023}

% Custom title page.
\defbeamertemplate*{title page}{customized}[1][]{
	\centering
	\usebeamerfont{title}\usebeamercolor[fg]{title}\inserttitle\par
	\usebeamerfont{subtitle}\usebeamercolor[fg]{subtitle}\insertsubtitle\par
	\bigskip
	\usebeamerfont{author}\usebeamercolor[fg]{normal text}\textbf{\insertauthor}\par
	\bigskip
	\usebeamerfont{date}\usebeamercolor[fg]{normal text}\textbf{\insertdate}\par
	\bigskip
	\bigskip

	\begin{columns}
		\begin{column}{.5\textwidth}
			\centering

			\includegraphics[scale=2]{microsoft.pdf}
		\end{column}
		\begin{column}{.5\textwidth}
			\centering

			\includegraphics[scale=.25]{inspektor_gadget.pdf}
		\end{column}
	\end{columns}
}

\begin{document}
	% Put these parameters here to avoid compilation error:
	% "! LaTeX Error: Missing \begin{document}."
	% Remove the navigation bar, this is useless...
	\setbeamertemplate{navigation symbols}{}
	% Use square instead of bubbles, see:
	% https://tex.stackexchange.com/a/69721
	\setbeamertemplate{section in toc}[square]
	% Modify the shaded value to 60% instead of 20%, see:
	% https://tex.stackexchange.com/a/66703
	\setbeamertemplate{sections/subsections in toc shaded}[default][50]
	% Use circle instead of bubbles for itemize, see:
	% \setbeamertemplate{itemize items}[circle]
	\setbeamertemplate{itemize items}[square]
	\setbeamertemplate{enumerate items}[square]

	\maketitle

	\section{Introduction}
	\begin{frame}
		\frametitle{eBPF, \texttt{libbpf} and \texttt{cilium/ebpf}}

		According to Brendan Gregg \cite{gregg_learn_2019}:
		\begin{quote}
			eBPF does to Linux what JavaScript does to HTML.
			[...]
			And with eBPF, [...], you can now write mini programs that run on events like disk I/O, which are run in a safe virtual machine in the kernel.
		\end{quote}

		\onslide<2->{
			\texttt{libbpf}, which is written in \texttt{C}, handles \cite{libbpf_contributors_libbpf_nodate}:
			\begin{enumerate}
				\item loading,
				\item verifying
				\item and attaching BPF programs to kernel hooks
			\end{enumerate}

			\texttt{cilium/ebpf} is the \texttt{golang} counterpart to \texttt{libbpf} \cite{cilium_contributors_ciliumebpf_nodate}.
		}
	\end{frame}

	\section{Problem}
	\begin{frame}[fragile]
		\frametitle{\texttt{\_\_kconfig}}

		An example in kernel source code \cite{song_selftestsbpf_2020}:
		\begin{verbatim}
			extern unsigned CONFIG_HZ __kconfig;

			/* ... */

			u64 tick_nsec = (NSEC_PER_SEC + CONFIG_HZ/2) / CONFIG_HZ;
		\end{verbatim}

		\onslide<2->{
			It is supported in:
			\begin{itemize}
				\item[\cmark] \texttt{libbpf}
				\item[\xmark] \texttt{cilium/ebpf}, you need workarounds:
				\begin{itemize}
					\item Adding a \texttt{bool} eBPF parameter.
					\item Using \texttt{#ifdef} and having two eBPF bytecodes.
				\end{itemize}
			\end{itemize}
		}
	\end{frame}

	\section{Solution}
	\begin{frame}
		\frametitle{Adding \texttt{\_\_kconfig} support to \texttt{cilium/ebpf}}

		\centering

		\includegraphics<1>[scale=3]{kconfig-fig1.pdf}%
		\includegraphics<2>[scale=3]{kconfig-fig2.pdf}%
		\includegraphics<3>[scale=3]{kconfig-fig3.pdf}%
		\includegraphics<4>[scale=3]{kconfig-fig4.pdf}%
		\includegraphics<5>[scale=3]{kconfig-fig5.pdf}%
		\includegraphics<6>[scale=3]{kconfig-fig6.pdf}%
		\includegraphics<7>[scale=3]{kconfig-fig7.pdf}%
		\includegraphics<8>[scale=3]{kconfig-fig8.pdf}%
	\end{frame}

	\begin{frame}[fragile, t]
		\frametitle{Example}

		The \texttt{profile block-io} gadget \cite{laniel_pkggadgets_2022, laniel_pkggadgets_2023}:

		\bigskip

		\begin{columns}[T]
			\begin{column}{.5\textwidth}
				Before:

				{\scriptsize
					\begin{verbatim}




						#ifdef KERNEL_BEFORE_5_11
						  return trace_rq_start((void *)ctx[1], false);
						#else /* !KERNEL_BEFORE_5_11 */
						  return trace_rq_start((void *)ctx[0], false);
						#endif /* !KERNEL_BEFORE_5_11 */
					\end{verbatim}
				}
			\end{column}
			\begin{column}{.5\textwidth}<2->
				After:

				{\scriptsize
					\begin{verbatim}
						extern int LINUX_KERNEL_VERSION __kconfig;

						/* ... */

						if (LINUX_KERNEL_VERSION < KERNEL_VERSION(5, 11, 0))
						  return trace_rq_start((void *)ctx[1], false);
						else
						  return trace_rq_start((void *)ctx[0], false);
					\end{verbatim}
				}
			\end{column}
		\end{columns}

	\end{frame}

	\section{Conclusion}
	\begin{frame}
		\frametitle{Conclusion}
		\begin{enumerate}
			\item \texttt{\_\_kconfig} support was added to \texttt{cilium/ebpf} \texttt{v0.11.0} \cite{cilium_contributors_ciliumebpf_nodate-1}.
			\item You can use it to modify eBPF programs behavior at runtime rather than using workarounds.
		\end{enumerate}

		\bigskip

		\textcolor{structure}{I would like to thank Lorenz Bauer and Timo Beckers for their suggestions and reviews!}
	\end{frame}

	\begin{frame}[allowframebreaks, noframenumbering]
		\frametitle{Bibliography}
		\setbeamertemplate{bibliography item}[text]

		\begin{scriptsize}
			\bibliographystyle{IEEEtran}
			\bibliography{beamer}
		\end{scriptsize}
	\end{frame}
\end{document}
