\newif \ifshare
% \sharetrue % Comment this if you want animation
\ifshare % "Share mode" without animation.
\documentclass[table, trans, aspectratio = 169]{beamer}
\else % "Presentation mode" with animation.
\documentclass[table, aspectratio = 169]{beamer}
\fi
\usepackage{default}
\usepackage[T1]{fontenc}
\usepackage{color}
\usepackage{graphicx}
\usepackage{subfig}
\usepackage{diagbox}
\usepackage{minted}
\usepackage{fancyvrb}


\let\pgfmathMod=\pgfmathmod\relax

\graphicspath{{../figs/PDF/}}

\usetheme{Boadilla}

\definecolor{struct}{HTML}{03a9f4}
\definecolor{alert}{HTML}{f44336}
\definecolor{example}{HTML}{aeea00}
\definecolor{good}{HTML}{8bc34a}
\definecolor{notgoodnotbad}{HTML}{ff9800}

\setbeamercolor{structure}{fg = struct}
\setbeamercolor{normal text}{fg = white, bg = black}
\setbeamercolor{example text}{fg = example}
\setbeamercolor{alerted text}{fg = alert}
\setbeamercolor{footline}{fg = white}

\title[eBPF summit 2022]{Porting eBPF CO-RE to arm64 Leads to Fix the Kernel}
\subtitle{eBPF summit 2022}
\author[Francis Laniel (\texttt{flaniel@linux.microsoft.com})]{Francis Laniel\\\texttt{flaniel@linux.microsoft.com}}
\date{29th September 2022}

% Custom title page.
\defbeamertemplate*{title page}{customized}[1][]{
	\centering
	\usebeamerfont{title}\usebeamercolor[fg]{title}\inserttitle\par
	\usebeamerfont{subtitle}\usebeamercolor[fg]{subtitle}\insertsubtitle\par
	\bigskip
	\usebeamerfont{author}\usebeamercolor[fg]{normal text}\textbf{\insertauthor}\par
	\bigskip
	\usebeamerfont{date}\usebeamercolor[fg]{normal text}\textbf{\insertdate}\par
	\bigskip
	\bigskip

	\begin{columns}
		\begin{column}{.5\textwidth}
			\centering

			\includegraphics[scale=2]{microsoft.pdf}
		\end{column}
		\begin{column}{.5\textwidth}
			\centering

			\includegraphics{kinvolk.pdf}
		\end{column}
	\end{columns}
}

\begin{document}
	% Put these parameters here to avoid compilation error:
	% "! LaTeX Error: Missing \begin{document}."
	% Remove the navigation bar, this is useless...
	\setbeamertemplate{navigation symbols}{}
	% Use square instead of bubbles, see:
	% https://tex.stackexchange.com/a/69721
	\setbeamertemplate{section in toc}[square]
	% Modify the shaded value to 60% instead of 20%, see:
	% https://tex.stackexchange.com/a/66703
	\setbeamertemplate{sections/subsections in toc shaded}[default][50]
	% Use circle instead of bubbles for itemize, see:
	% \setbeamertemplate{itemize items}[circle]
	\setbeamertemplate{itemize items}[square]
	\setbeamertemplate{enumerate items}[square]

	\maketitle

	\section{\texttt{bpf2go}}
	\begin{frame}
		\frametitle{\texttt{bpf2go}}
		\framesubtitle{What it is?}

		\texttt{bpf2go} enables you to \cite{cilium_contributors_bpf2go_nodate}:
		\begin{quote}
			[compile] a C source file into eBPF bytecode and then emits a Go file containing the eBPF.
		\end{quote}

		\bigskip

		\onslide<2>{
			It permits you to generate eBPF bytecode for several architectures, by calling \texttt{clang}, like \cite{cilium_contributors_bpf2go_nodate-1}:
			\begin{itemize}
				\item \texttt{amd64}
				\item \textcolor{struct}{\texttt{arm64}}
			\end{itemize}
		}
	\end{frame}

	\section{Problem and context}
	\begin{frame}[fragile]
		\frametitle{Problem and context}

		\begin{columns}
			\begin{column}{.5\textwidth}
				\centering

				\includegraphics[scale = .33]{inspektor_gadget.pdf}
			\end{column}
			\begin{column}{.5\textwidth}
				\centering

				\includegraphics[scale = .33]{arm.pdf}
			\end{column}
		\end{columns}

		\bigskip

		\begin{onlyenv}<2->
			\centering

			\begin{verbatim}
# Francis uses execsnoop!
$ ./kubectl-gadget trace exec -A
NODE NAMESPACE POD CONTAINER PID PPID PCOMM RET ARGS
# But nothing happened!
			\end{verbatim}
		\end{onlyenv}
	\end{frame}

	\section{Contribution}
	\begin{frame}
		\frametitle{Contribution}
		\framesubtitle{Investigate the root cause}

		\centering

		% Only sys_exit is not reported.
		% The problem occurs with execsnoop bcc, perf, etc.

		\includegraphics<1>[scale=.33]{exec_trace_calls-fig1.pdf}%
		\includegraphics<2>[scale=.33]{exec_trace_calls-fig2.pdf}%
		\includegraphics<3>[scale=.33]{exec_trace_calls-fig3.pdf}%
	\end{frame}

	\begin{frame}[fragile]
		\frametitle{Contribution}
		\framesubtitle{Upstream bug fix and workaround for older kernels}

		\begin{columns}
			\begin{column}{.5\textwidth}
				\centering
				Upstream fix \cite{linux_kernel_contributors_arm64_2022}:
				\begin{tiny}
					\begin{minted}[breaklines]{diff}
--- a/arch/arm64/include/asm/processor.h
+++ b/arch/arm64/include/asm/processor.h
@@ -272,8 +272,9 @@ void tls_preserve_current_state(void);

 static inline void start_thread_common(struct pt_regs *regs, unsigned long pc)
 {
+	s32 previous_syscall = regs->syscallno;
 	memset(regs, 0, sizeof(*regs));
-	forget_syscall(regs);
+	regs->syscallno = previous_syscall;
 	regs->pc = pc;

 	if (system_uses_irq_prio_masking())
					\end{minted}
				\end{tiny}
			\end{column}
			\begin{column}{.5\textwidth}<2->
				\centering
				Use \texttt{kprobe} for older kernels \cite{inspektor_gadget_contributors_pkggadgets_2022}:
				\begin{itemize}
					\item[\checkmark] Permits running \texttt{execsnoop}
					\item[$\times$] Arguments will not be traced
				\end{itemize}
			\end{column}
		\end{columns}

	\end{frame}

	\section{Conclusion and future work}
	\begin{frame}
		\frametitle{Conclusion and future work}

		Conclusion:
		\begin{enumerate}
			\item Kernel bug was fixed and backported to stable ones, so you can now trace \texttt{execve} syscall family on \texttt{arm64} \cite{linux_kernel_contributors_arm64_2022, levin_patch_2022}.
			\item Inspektor Gadget was ported on \texttt{arm64} \cite{inspektor_gadget_contributors_add_2022}.
		\end{enumerate}

		\bigskip

		\onslide<2->{
			Future work:
			\begin{enumerate}
				\item Test Inspektor Gadget \texttt{arm64} port on several platforms.
			\end{enumerate}
		}

		\bigskip

		\onslide<3->{
			Thanks to Jeremi Piotrowski for his help finding this kernel bug!
		}
	\end{frame}

	\begin{frame}[allowframebreaks, noframenumbering]
		\frametitle{Bibliography}
		\setbeamertemplate{bibliography item}[text]

		\begin{scriptsize}
			\bibliographystyle{IEEEtran}
			\bibliography{beamer}
		\end{scriptsize}
	\end{frame}
\end{document}
